\section{Task 5: Comparison between Single-processor and Multi-processor}

For the ranges defined in the task, the Single-Processor and the Multi-Processor-solutions
have the same results ($k = 4, ... 14$), however, when we try to increase k further, we
notice a difference between the solutions. The Single-Processor-solution has the following results
for $k = 4, ... 24$:
\inputminted{matlab}{singleout}

These are quite similar to the results for 4 and 16 Processors, however there are a few differences,
notably 4 Processors gets these differing results:

\inputminted{matlab}{4proc}

and 16 Processors gets a minor difference again in the higher cases:

\inputminted{matlab}{16proc}

Only the differences from the single-processors solution is shown here.

The variance in results here stems from the fact that the Single-Processor solution does a linear sum
of elements, and will thus have a higher amount of accumulated floating point error, as small numbers
will be ignored due to round-off-error when the accumulated sum gets large enough. The Multi-Processor
solution on the other hand, does partial sums that are then accumulated together, which makes the partial
sums smaller, and will thus accumulate more of the small partial sums before losing them to round-off-errors,
it thus also follows that the more processors that does partial sums, the more partial sums we split the
summation into, and thus we will have smaller partial sums, allowing us to accumulate more of the sums before
losing any of the partial sums to round-off error.

The answer should thus NOT be the same for all these cases, but in fact better with larger P.

