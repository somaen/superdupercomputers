\section{Utilities, Compilers and Hardware}

\subsection{Libraries}
Our program needs to be linked against the following libraries
\begin{itemize}
    \item{\texttt{libmpi}}  -  to get MPI-functionality
    \item{\texttt{libm}}  -  to get the necessary math-functionality.
\end{itemize}

\subsection{Compilers}
On Kongull we used the following compilers to compile our program:
\begin{itemize}
    \item{\texttt{GCC 4.7.0}}
    \item{\texttt{GFortran 4.7.0}}
\end{itemize}

Both of which had to be activated explicitly with \texttt{module load gcc/4.7.0}.

Locally we also did some testing, using these compiler-versions, with all warnings enabled to make sure that our code was as well written and portable as possible:
\begin{itemize}
    \item{\texttt{GCC 4.7.2}}
    \item{\texttt{GCC 4.2.1 (Apple LLVM)}}
    \item{\texttt{Clang 3.0}}
    \item{\texttt{Clang Apple LLVM 4.2 (based on LLVM 3.2svn)}}
    \item{\texttt{Gfortran 4.5.4}}
    \item{\texttt{Gfortran 4.7.2}}
\end{itemize}

\subsection{Kongull}

The distributed memory computer we ran our program on is called Kongull, it consists of 93 compute nodes that each have the following specs\footnote{\url{http://docs.notur.no/Members/hrn/kongull.hpc.ntnu.no/kongull-hardware-1/hardware}}:
\begin{itemize}
    \item 24/48 GiB RAM
    \item 2x6 Core AMD Opteron 2431, each with the following specs:
    \begin{itemize}
        \item 2.4 GHz
        \item 6 x 128 L1 Cache
        \item 6 x 512 KiB L2 Cache
        \item 6 MiB L3 Cache
    \end{itemize}
\end{itemize}

Each of these nodes has 2 CPUs, that share memory, thus we enjoy the combination of shared memory computing AND distributed memory while running on this cluster.

As a side note, the hardware-listings we read listed the machine as being homogenous, while our results differed enough that we are led to believe that Kongull is atleast somewhat heterogenous in it's CPU-setup.