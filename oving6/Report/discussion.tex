\section{Discussion}

\TODO{a discussion of bottlenecks and possible optimization}

Our program seems to have some variance in it's runtime, timing the various processes show values
between 0.5 and 9 seconds, this stems from the fact that the processes are using blocking communication
for the transpose, so the slowest process will delay the faster processes at these points, effectively creating
2 bottlenecks in our runs (one for each of the matrix transposes).

Our program is dominated by the (inverse) fast-sine-transform-calls runtime-wise, which we have little possibility
to optimise any further. We also use a large amount of small functions to do the various tasks that are required to
set up and manage the matrix, however, none of these are close to any tight loops, and most of them will be optimised
by the compiler so that inlining this functionality would not create any further runtime-benefits, but instead make the
code base less manageable.

The decision to wrap the fortran-code into C-functions should also not have any effect on the runtime, for the same reason
explained above, the functions will be link-time-optimised afterwards anyhow.
