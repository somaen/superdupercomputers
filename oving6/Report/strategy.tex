\section{Solution strategies}
Given that we have a supercomputer at our hands we could be tempted to just
throw processing power on the problem until it disappears, but a simple solver
based on Elementary linear algebra would be very slow compared to other
solutions. A solver using gaussian elimination or LU-factorisation would have a
vector of unknowns being \emph{n}$^2$ long and require \emph{n}$^6$ floating
point operations. Given that each unknown only depends on its neighbours we have
a banded matrix and can shave off \emph{n}$^2$ and achieve the same result in
\emph{n}$^4$. 

Using properties that is in our stencil, our discretisation of the Laplace
operator, is itself transposed so we can use a three-point stencil: T, $\mathrm{PU}=\mathrm{TU+UT}$, which is
the same as we use in two dimensions, both ways, just swapping what side we
multiply with, then diagonalising T, putting us at \emph{n}$3$ floating point
operations. A further improvement on this method can be done replacing a vector
matrix product with a Fast sine transform, a transpose and a fast inverse sine
transform landing us at $\log(n)\cdot n^2$. We are not attempting to further
explain as we did not quite understand the diagonalisation part. 
	
